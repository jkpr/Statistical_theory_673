% /**
%  * A template for homework files in math classes. The 
%  * packages and newcommands are a good starting point.
%  *
%  * Author: James K. Pringle
%  * E-mail: jameskpringle@gmail.com
%  * Last Changed: 5 September 2013
%  *
%  * "LaTeX countains the increasing union of MS Word"
%  */
%~~~~~~~~~~~~~~~~~~~~~~~~~~~~~~~~~~~~~~~~~~~~~~~~~~~~~~~~~%
%%%%%%%%%%%%%%%%%%%%%%%%%%%%%%%%%%%%%%%%%%%%%%%%%%%%%%%%%%%
%                                                         %
%                        PAGE SETUP                       %
%                                                         %
%%%%%%%%%%%%%%%%%%%%%%%%%%%%%%%%%%%%%%%%%%%%%%%%%%%%%%%%%%%
\documentclass[letterpaper, 12pt]{article}\usepackage[]{graphicx}\usepackage[]{color}
%% maxwidth is the original width if it is less than linewidth
%% otherwise use linewidth (to make sure the graphics do not exceed the margin)
\makeatletter
\def\maxwidth{ %
  \ifdim\Gin@nat@width>\linewidth
    \linewidth
  \else
    \Gin@nat@width
  \fi
}
\makeatother

\definecolor{fgcolor}{rgb}{0.345, 0.345, 0.345}
\newcommand{\hlnum}[1]{\textcolor[rgb]{0.686,0.059,0.569}{#1}}%
\newcommand{\hlstr}[1]{\textcolor[rgb]{0.192,0.494,0.8}{#1}}%
\newcommand{\hlcom}[1]{\textcolor[rgb]{0.678,0.584,0.686}{\textit{#1}}}%
\newcommand{\hlopt}[1]{\textcolor[rgb]{0,0,0}{#1}}%
\newcommand{\hlstd}[1]{\textcolor[rgb]{0.345,0.345,0.345}{#1}}%
\newcommand{\hlkwa}[1]{\textcolor[rgb]{0.161,0.373,0.58}{\textbf{#1}}}%
\newcommand{\hlkwb}[1]{\textcolor[rgb]{0.69,0.353,0.396}{#1}}%
\newcommand{\hlkwc}[1]{\textcolor[rgb]{0.333,0.667,0.333}{#1}}%
\newcommand{\hlkwd}[1]{\textcolor[rgb]{0.737,0.353,0.396}{\textbf{#1}}}%

\usepackage{framed}
\makeatletter
\newenvironment{kframe}{%
 \def\at@end@of@kframe{}%
 \ifinner\ifhmode%
  \def\at@end@of@kframe{\end{minipage}}%
  \begin{minipage}{\columnwidth}%
 \fi\fi%
 \def\FrameCommand##1{\hskip\@totalleftmargin \hskip-\fboxsep
 \colorbox{shadecolor}{##1}\hskip-\fboxsep
     % There is no \\@totalrightmargin, so:
     \hskip-\linewidth \hskip-\@totalleftmargin \hskip\columnwidth}%
 \MakeFramed {\advance\hsize-\width
   \@totalleftmargin\z@ \linewidth\hsize
   \@setminipage}}%
 {\par\unskip\endMakeFramed%
 \at@end@of@kframe}
\makeatother

\definecolor{shadecolor}{rgb}{.97, .97, .97}
\definecolor{messagecolor}{rgb}{0, 0, 0}
\definecolor{warningcolor}{rgb}{1, 0, 1}
\definecolor{errorcolor}{rgb}{1, 0, 0}
\newenvironment{knitrout}{}{} % an empty environment to be redefined in TeX

\usepackage{alltt}

% 1in margins all the way around
\usepackage[margin=1in]{geometry}

% Sets \parindent to 0 and \parskip to stretchable.
\usepackage{parskip}
% Use for bigger spaces between paragraphs.
%\parskip=1.5\baselineskip

% Set headers and footers
\usepackage{fancyhdr}
\pagestyle{fancy}
% Header
\renewcommand{\headrulewidth}{0.4pt}
\lhead{\textsc{\mathclass}}
\chead{\textsc{\today}}
\rhead{\textsc{\mynamehdr}}
% Footer
\renewcommand{\footrulewidth}{0.4pt}
\lfoot{}
\cfoot{\thepage}
\rfoot{}

% Make the space between lines slightly more generous 
% than normal single spacing, but compensate so that the 
% spacing between rows of matrices still looks normal.  
% Note that 1.1=1/.9090909...
\renewcommand{\baselinestretch}{1.1}
\renewcommand{\arraystretch}{.91}

%%%%%%%%%%%%%%%%%%%%%%%%%%%%%%%%%%%%%%%%%%%%%%%%%%%%%%%%%%%
%                                                         %
%                      USEFUL PACKAGES                    %
%                                                         %
%%%%%%%%%%%%%%%%%%%%%%%%%%%%%%%%%%%%%%%%%%%%%%%%%%%%%%%%%%%

% The classic three
\usepackage{amsmath,amsthm,amssymb}

% Define \newtheorem for use
% No numbers, labeled 'Theorem'
\newtheorem*{nthm}{Theorem}

% Not sure what this is for
\usepackage{amsfonts}

% Fancy script font
\usepackage{mathrsfs}

% Makes enumerate environment much easier to customize
% by specifying the counter
\usepackage{enumerate}

% Color
\usepackage{color}
\usepackage[usenames,dvipsnames,svgnames,table]{xcolor}

% URL links
\usepackage{hyperref}

% For inserting graphics and images
\usepackage{graphicx}
\usepackage{float}
\usepackage[footnotesize]{caption}



%%%%%%%%%%%%%%%%%%%%%%%%%%%%%%%%%%%%%%%%%%%%%%%%%%%%%%%%%%%
%                                                         %
%                   USER-DEFINED COMMANDS                 %
%                                                         %
%%%%%%%%%%%%%%%%%%%%%%%%%%%%%%%%%%%%%%%%%%%%%%%%%%%%%%%%%%%

% Make a hyperlink with colored text
\newcommand{\hrefcolor}[3]{\href{#1}{\textcolor{#3}{#2}}}

% Make a hyperlink with gray text
\newcommand{\hrefgray}[2]{\hrefcolor{#1}{#2}{Gray}}

% Make the header for the first page
\newcommand{\firstpageinfo}{
\textsf{
\begin{flushleft}
\sc \myname \\
\normalfont \mathclass \\
\professorname \\
\assignmentnumber \\
\thedate
\end{flushleft}
} \bigskip
}

% Make problem list for "title" of page
\newcommand{\problemlist}{ 
\begin{center}
\textbf{\Large \textsf{\assignmentnumber}}\\
\textit{\textsf{\problemset}}
\end{center}
\bigskip
}

%~~~~~~~~~~~~~~~~~~~~~~~~~~~~~~~~~~~~~~~~~~~~~~~~~~~~~~~~~%
%                                                         %
%               LETTERS, FUNCTIONS, AND TEXT              %
%                                                         %
%~~~~~~~~~~~~~~~~~~~~~~~~~~~~~~~~~~~~~~~~~~~~~~~~~~~~~~~~~%

% A
\newcommand{\cA}{\mathcal{A}}
\newcommand{\sA}{\mathscr{A}}
\renewcommand{\aa}{\;\text{a.a.}}
\renewcommand{\ae}{\;\text{a.e.}}
% B
\newcommand{\B}{\mathscr{B}}
\newcommand{\cB}{\mathcal{B}}
% C
\newcommand{\cC}{\mathcal{C}}
\newcommand{\cov}{\text{cov}}
% E
\newcommand{\E}{\mathbb{E}}
% F
\newcommand{\sF}{\mathscr{F}}
\newcommand{\cF}{\mathcal{F}}
\newcommand{\Ft}{F^\sim}
% G
\newcommand{\cG}{\mathcal{G}}
\newcommand{\sG}{\mathscr{G}}
% I
\newcommand{\io}{\;\text{i.o.}}
% N
\newcommand{\N}{\mathbb{N}}
% P
\newcommand{\cP}{\mathcal{P}}
\newcommand{\sP}{\mathscr{P}}
\newcommand{\pr}{\text{pr}}
% Q
\newcommand{\Q}{\mathbb{Q}}
% R
\newcommand{\R}{\mathbb{R}}
\newcommand{\bR}{\mathbf{R}}
\newcommand{\cR}{\mathcal{R}}
% S
\newcommand{\cS}{\mathcal{S}}
% U
\newcommand{\cU}{\mathcal{U}}
% V
\newcommand{\var}{\text{var}}
% Z
\newcommand{\Z}{\mathbb{Z}}
% Punctuation
\newcommand{\sbs}{\;|\;} % space bar space
% Math
\newcommand{\imii}{\int_{-\infty}^\infty}
\newcommand{\pion}{\prod_{i=1}^n}
\newcommand{\pioI}{\prod_{i=1}^I}
\newcommand{\pjon}{\prod_{j=1}^n}
\newcommand{\pjoJ}{\prod_{j=1}^J}
\newcommand{\pkon}{\prod_{k=1}^n}
\newcommand{\pkoK}{\prod_{k=1}^K}
\newcommand{\sion}{\sum_{i=1}^n}
\newcommand{\sioI}{\sum_{i=1}^I}
\newcommand{\sjon}{\sum_{j=1}^n}
\newcommand{\sjoJ}{\sum_{j=1}^J}
\newcommand{\skon}{\sum_{k=1}^n}
\newcommand{\skoK}{\sum_{k=1}^K}
\newcommand{\sioi}{\sum_{i=1}^\infty}
\newcommand{\sjoi}{\sum_{j=1}^\infty}
\newcommand{\skoi}{\sum_{k=1}^\infty}
\newcommand{\sio}{\sum_{i=1}}
\newcommand{\sjo}{\sum_{j=1}}
\newcommand{\sko}{\sum_{k=1}}
% Typography
\newcommand{\scb}[1]{\textsc{\textbf{#1}}}

%~~~~~~~~~~~~~~~~~~~~~~~~~~~~~~~~~~~~~~~~~~~~~~~~~~~~~~~~~%
%                                                         %
%            CHANGE THESE BASED ON THE PAPER              %
%                                                         %
%~~~~~~~~~~~~~~~~~~~~~~~~~~~~~~~~~~~~~~~~~~~~~~~~~~~~~~~~~%

% Constants for fancy header and first page info
\newcommand{\mynamehdr}{\hrefgray{http://biostat.jhsph.edu/~jpringle/}{\myname}}
\newcommand{\mathclass}{140.673 Stat Theory}
\newcommand{\myname}{James K. Pringle}
\newcommand{\professorname}{Dr. Constantine Frangakis}
\newcommand{\assignmentnumber}{Problem set 1}
\newcommand{\thedate}{\today}
\newcommand{\problemset}{Decision theory and loss functions}

%%%%%%%%%%%%%%%%%%%%%%%%%%%%%%%%%%%%%%%%%%%%%%%%%%%%%%%%%%%
%                                                         %
%                      BEGIN DOCUMENT                     %
%                                                         %
%%%%%%%%%%%%%%%%%%%%%%%%%%%%%%%%%%%%%%%%%%%%%%%%%%%%%%%%%%%
\IfFileExists{upquote.sty}{\usepackage{upquote}}{}
\begin{document}

% Take header off of first page
\thispagestyle{empty}

% Put in first page info (top of page)
\firstpageinfo

% Put in title for the paper
\problemlist

\subsection*{Problem 1}
Among the population, $P$, of women who visit phyisicians for screening for a disease, assume that the screening test has specificity and sensitivity as the one discussed in lecture one, where, here, probability statements mean fractions of women in the population $P$ (for example, specificity of $98\%$ means that, of all true negative women in $P$, $98\%$ would test negative.).
For a woman $i$, denote $\theta(i) = 1$ if the woman is truly positive, and $2$ if truly negative; and denote $(l_i(a_1), l_i(a_2))$ to be the loss to that woman if treated, or if not treated, respectively (in the last expressions, her true status is captured already in the notation ``i'').
Suppose that the averages of the losses in the diseased and non-diseased women, if treated and if not treated, are:
\begin{align*}
l(1,a_1) &:= E\{l_i(a_1)| \theta(i) = 1\} = 2 \\
l(1, a_2) &:= E\{l_i(a_2)|\theta(i)=1\} = 5 \\
l(2, a_1) &:= E\{l_i(a_1)|\theta(i)=2\} = 1 \\
l(2, a_2) &:= E\{l_i(a_2)|\theta(i)=2\} = 0
\end{align*}
but that $l_i$ may generally vary from woman to woman, and that among women of a particular status (diseased, or not diseased), the losses $l_i$ may be correlated with the value $X_i$ that the diagnostic test would show for woman $i$.
For the strategy $s$ defined as $s(X_i) = a_1$ if $X_i$ is positive, and $s(X_i)=a_2$ if $X_i$ is negative, which of the following conditions 1.-3. would make the losses
\begin{equation} 
\label{eq1}
E\{l_i(s(X_i)) | \theta(i)\} \text{ and } E\{l(\theta(i),s(X_i))| \theta(i)\}
\end{equation}
equal and why?
\begin{enumerate}[1.]
\item
For a fixed action $a$, $l_i(a)$ is constant within women of common disease status $\theta(i)$.
\item
For a fixed action $a$, $X_i$ is independent 	of $l_i(a)$.
\item
For a fixed action $a$, $X_i$ is independent of $l_i(a)$ given $\theta(i)$.
\end{enumerate}

\begin{proof}
If $X_i \geq 0$, then $s(X_i) = a_1$ and if $X_i < 0$, then $s(X_i)=a_2$. Thus,
\[
l(\theta(i), s(X_i)|\theta_i) = l(\theta(i), a_1) I_{X_i \geq 0} + l(\theta(i), a_2) I_{X_i < 0}
\]
Hence, the right-hand side becomes
\begin{align}
E\{l(\theta(i), s(X_i)|\theta_i) \sbs \theta(i)\} 
&= 
E\{l(\theta(i), a_1) I_{X_i \geq 0} + l(\theta(i), a_2) I_{X_i < 0} \sbs \theta(i)\}
\\
&=
E\{l(\theta(i), a_1) I_{X_i \geq 0} \sbs \theta(i)\} +
E\{l(\theta(i), a_2) I_{X_i < 0} \sbs \theta(i)\}
\end{align}
Since the expression above is conditioned on $\theta(i)$, the expressions $l(\theta(i), a_1)$ and $l(\theta(i), a_2)$ are constant by hypothesis. The right-hand side can be further simplified to
\begin{equation}
%E\{l(\theta(i), s(X_i)|\theta_i) \sbs \theta(i)\}  =
l(\theta(i), a_1)E\{I_{X_i \geq 0} \sbs \theta(i)\} +
l(\theta(i), a_2)E\{ I_{X_i < 0} \sbs \theta(i)\}
\label{rhsSimp}
\end{equation}

Similarly, the left-hand side can be written as
\begin{align}
E\{l_i(s(X_i)) \sbs \theta(i)\} &=
E\{l_i(a_1)I_{X_i \geq 0} + l_i(a_2) I_{X_i < 0} \sbs \theta(i)\}
\\
&=
E\{l_i(a_1)I_{X_i \geq 0} \sbs \theta(i)\}
 + E\{l_i(a_2) I_{X_i < 0} \sbs \theta(i)\}
\label{lhs}
\end{align}
Examine condition 1. For fixed $a$, Let $l_i(a)$ be constant within women of common disease status $\theta(i)$. The expectation of a constant function is that same constant, hence 
\begin{equation}
E\{l_i(a) \sbs \theta(i)=j\} = l_i(a) |_{\theta(i)=j}
\label{const}
\end{equation}
Combining \eqref{lhs} with \eqref{const}
\begin{align}
E\{l_i(s(X_i)) \sbs \theta(i)\} &=
E\{l_i(a_1)I_{X_i \geq 0} \sbs \theta(i)\}
 + E\{l_i(a_2) I_{X_i < 0} \sbs \theta(i)\}
\\
&=
E\{E\{l_i(a_1) \sbs \theta(i)\}I_{X_i \geq 0} \sbs \theta(i)\}
 + 
E\{E\{l_i(a_2) \sbs \theta(i)\}I_{X_i < 0} \sbs \theta(i)\}
\\
&=
E\{l_i(a_1) \sbs \theta(i)\} E\{I_{X_i \geq 0} \sbs \theta(i)\}
+
E\{l_i(a_2) \sbs \theta(i)\} E\{I_{X_i < 0} \sbs \theta(i)\}
\\
&=
l(\theta(i), a_1)E\{I_{X_i \geq 0} \sbs \theta(i)\}
+
l(\theta(i), a_2)E\{I_{X_i < 0} \sbs \theta(i)\}
\label{constSimp}
\end{align}
Clearly, \eqref{rhsSimp} and \eqref{constSimp} are the same, and therefore the right- and left-hand sides of \eqref{eq1} are equal given condition one.

Given condition three, for a fixed action $a$, $X_i$ is independent of $l_i(a)$ given $\theta(i)$. Therefore the exents in $\sigma(X_i)$ and $\sigma(l_i(a))$ are independent given $\theta(i)$. 
Since the expectation of independent events is the production of the expectation of those events, from \eqref{lhs} it follows that
\begin{align}
E\{l_i(s(X_i)) \sbs \theta(i)\}
&=
E\{l_i(a_1)I_{X_i \geq 0} \sbs \theta(i)\}
+ 
E\{l_i(a_2) I_{X_i < 0} \sbs \theta(i)\}
\\
&=
E\{l_i(a_1)\sbs \theta(i)\} E\{I_{X_i \geq 0} \sbs \theta(i)\}
+
E\{l_i(a_2)\sbs \theta(i)\} E\{I_{X_i < 0} \sbs \theta(i)\}
\\
&=
l(\theta(i), a_1)E\{I_{X_i \geq 0} \sbs \theta(i)\}
+
l(\theta(i), a_2)E\{I_{X_i < 0} \sbs \theta(i)\}
\label{condIndSimp}
\end{align}
Clearly, \eqref{rhsSimp} and \eqref{condIndSimp} are the same, and therefore the right- and left-hand sides of \eqref{eq1} are equal given condition three.

Finally, condition 2 says that for a fixed action $a$, $X_i$ is independent of $l_i(a)$. This however does not imply that conditional on the event $\theta(i)$ that $X_i$ is independent of $l_i(a)$. 
%http://en.wikipedia.org/wiki/Conditional_independence
For a counterexample, let $\cup_{i=1}^4 A_i = \Omega$ and let the $A_i$ be disjoint.

\begin{table}[H]
\centering
\begin{tabular}{c c c c c}
\hline\hline \\[-2.0ex]
Event & $P$ & X & Y & Z \\ [0.25ex]
\hline \\[-2.0ex]
$A_1$ & $P(A_1)=0.25$ & $X=0$ & $Y=0$ & $Z=0$
\\
$A_2$ & $P(A_2)=0.25$ & $X=1$ & $Y=0$ & $Z=0$
\\
$A_3$ & $P(A_3)=0.25$ & $X=0$ & $Y=1$ & $Z=0$
\\
$A_4$ & $P(A_4)=0.25$ & $X=1$ & $Y=1$ & $Z=1$
\end{tabular}
\end{table}
Clearly, $X$ and $Y$ are independent. However, conditional on $Z=0$, 
\[
0 = P(X=1 \cap Y=1 \sbs Z=0) \neq P(X=1 \sbs Z=0) P(Y=1 \sbs Z=0) = 1/3^2 = 1/9
\]
Therefore, $X$ and $Y$ are not conditionally independent (on $Z$), and it is shown that independence does not in general imply conditional independence. 
Hence,
\begin{align}
E\{l_i(s(X_i)) \sbs \theta(i)\}
&=
E\{l_i(a_1)I_{X_i \geq 0} \sbs \theta(i)\}
+ 
E\{l_i(a_2) I_{X_i < 0} \sbs \theta(i)\}
\\
&\neq
E\{l_i(a_1)\sbs \theta(i)\} E\{I_{X_i \geq 0} \sbs \theta(i)\}
+
E\{l_i(a_2)\sbs \theta(i)\} E\{I_{X_i < 0} \sbs \theta(i)\}
\\
&=
l(\theta(i), a_1)E\{I_{X_i \geq 0} \sbs \theta(i)\}
+
l(\theta(i), a_2)E\{I_{X_i < 0} \sbs \theta(i)\}
\end{align}
Therefore the right- and left-hand sides of \eqref{eq1} are not in general equal given condition two.
\end{proof}

\subsection*{Problem 2}
Now assume that the way the test $X_i$ is determined is by measuring a continuous variable $X_i^*$, and calling $X_i$ positive if $X^* > 0$, otherwise calling $X_i$ negative. Assuming that pr$(X_i^* | \theta(i))$ is normal with variance $1$, find $E(X_i^*|\theta(i))$ for the two disease conditions. 
Also, assume that, for a fixed action $a$, pr$(l_i(a) | \theta(i))$ is normal with the means given above, variance $10$, and that $cor(l_i(a),X_i^*|\theta(i))=0.7$.
 Using simulation of $1000$ 	diseased and $1000$ non-diseased women, or otherwise, estimate the two average losses in \eqref{eq1}.

\begin{proof}
From the lecture notes, the specificiy of the test is $0.98$ and the sensitivity is $0.94$. Thus $P( X_i^* > 0 | \theta(i) = 1) = 0.94$. Since we assume a normal distribution, with variance $\sigma^2 = 1$, we have
\begin{align*}
0.94 &= P( X^* > 0 | \theta(i) = 1) \\
0.94 &= P( (X^* - \mu_1) / \sigma > -\mu_1/ \sigma | \theta(i) = 1) \\
0.94 &= P( Z > -\mu_1 | \theta(i) = 1)
\end{align*}
Thus we solve for $\mu_1 = E(X_i^* | \theta(i)=1)$ in $1 - \Phi(-\mu_1) = 0.94$ where $\Phi$ is the distribution function for the standard normal distribution. It follows that $\mu_1 = -\Phi^{-1} (0.06) = - \texttt{qnorm}(0.06) = 1.5548$. 

Similarly solving for $\mu_2 = E(X_i^* | \theta(i) = 2)$, one finds that
\begin{align*}
0.98 &= P(X^* < 0 | \theta(i) = 2) \\
0.98 &= P((X^* - \mu_2) \ \sigma < -\mu_2 / \sigma | \theta(i) = 2) \\
0.98 &= P( Z < -\mu_2 | \theta(i) = 2)
\end{align*}
So $\mu_2 = -\Phi^{-1}(0.98) = -\texttt{qnorm}(0.98) = -2.0537$.

\textbf{Simulation}

The algorithm is to 
\begin{enumerate}[(1)]
\item
Choose $\theta= 1$ and $n$, the number of simulated scores
\item
Generate $n$ of $X_i^*$, the distribution of which depends on $\theta$.
\item
Generate $l_i(s(X_i))$, the distribution of which depends on $X_i^*$ and $\theta$, using bivariate normal conditional distribution. Average these results.
\item
Generate $l(\theta(i), s(X_i^*))$, the distribution of which has two values given $\theta$ and the value depends on $X_i$. Average these results.
\item
Set $\theta = 2$ and $n=n$, go to (2)

\end{enumerate}

\begin{knitrout}
\definecolor{shadecolor}{rgb}{0.969, 0.969, 0.969}\color{fgcolor}\begin{kframe}
\begin{alltt}
\hlkwd{library}\hlstd{(xtable)}

\hlkwd{set.seed}\hlstd{(}\hlnum{2014} \hlopt{-} \hlnum{2} \hlopt{-} \hlnum{5}\hlstd{)}

\hlcom{# Generate a random l(a) | X* = x}
\hlstd{r_cond} \hlkwb{<-} \hlkwa{function}\hlstd{(}\hlkwc{x}\hlstd{,} \hlkwc{theta} \hlstd{=} \hlnum{1}\hlstd{) \{}
    \hlstd{rho} \hlkwb{<-} \hlnum{0.7}
    \hlstd{sigma_l} \hlkwb{<-} \hlkwd{sqrt}\hlstd{(}\hlnum{10}\hlstd{)}
    \hlstd{sigma_x} \hlkwb{<-} \hlnum{1}

    \hlcom{# Mean of l(a1) and l(a2) given a theta value}
    \hlstd{mu_a1} \hlkwb{<-} \hlnum{2}
    \hlstd{mu_a2} \hlkwb{<-} \hlnum{5}
    \hlkwa{if} \hlstd{(theta} \hlopt{==} \hlnum{2}\hlstd{) \{}
        \hlstd{mu_a1} \hlkwb{<-} \hlnum{1}
        \hlstd{mu_a2} \hlkwb{<-} \hlnum{0}
    \hlstd{\}}
    \hlstd{mu_l} \hlkwb{<-} \hlkwd{ifelse}\hlstd{(x} \hlopt{>=} \hlnum{0}\hlstd{, mu_a1, mu_a2)}
    \hlcom{# Mean of X* given a theta value}
    \hlstd{mu_x} \hlkwb{<-} \hlkwd{ifelse}\hlstd{(theta} \hlopt{==} \hlnum{1}\hlstd{,} \hlopt{-}\hlkwd{qnorm}\hlstd{(}\hlnum{0.06}\hlstd{),} \hlopt{-}\hlkwd{qnorm}\hlstd{(}\hlnum{0.98}\hlstd{))}

    \hlstd{new_mean} \hlkwb{<-} \hlstd{mu_l} \hlopt{+} \hlstd{sigma_l}\hlopt{/}\hlstd{sigma_x} \hlopt{*} \hlstd{rho} \hlopt{*} \hlstd{(x} \hlopt{-} \hlstd{mu_x)}
    \hlstd{new_var} \hlkwb{<-} \hlstd{(}\hlnum{1} \hlopt{-} \hlstd{rho}\hlopt{^}\hlnum{2}\hlstd{)} \hlopt{*} \hlstd{sigma_l}\hlopt{^}\hlnum{2}
    \hlstd{new_sd} \hlkwb{<-} \hlkwd{sqrt}\hlstd{(new_var)}
    \hlstd{val} \hlkwb{<-} \hlkwd{rnorm}\hlstd{(}\hlkwd{length}\hlstd{(x), new_mean, new_sd)}
    \hlkwd{return}\hlstd{(val)}
\hlstd{\}}

\hlstd{r_cond_ave} \hlkwb{<-} \hlkwa{function}\hlstd{(}\hlkwc{x}\hlstd{,} \hlkwc{theta} \hlstd{=} \hlnum{1}\hlstd{) \{}
    \hlkwd{mean}\hlstd{(}\hlkwd{r_cond}\hlstd{(x, theta))}
\hlstd{\}}

\hlstd{mean_loss} \hlkwb{<-} \hlkwa{function}\hlstd{(}\hlkwc{x}\hlstd{,} \hlkwc{theta} \hlstd{=} \hlnum{1}\hlstd{) \{}
    \hlstd{mu_a1} \hlkwb{<-} \hlnum{2}
    \hlstd{mu_a2} \hlkwb{<-} \hlnum{5}
    \hlkwa{if} \hlstd{(theta} \hlopt{==} \hlnum{2}\hlstd{) \{}
        \hlstd{mu_a1} \hlkwb{<-} \hlnum{1}
        \hlstd{mu_a2} \hlkwb{<-} \hlnum{0}
    \hlstd{\}}
    \hlkwd{return}\hlstd{(}\hlkwd{ifelse}\hlstd{(x} \hlopt{>=} \hlnum{0}\hlstd{, mu_a1, mu_a2))}
\hlstd{\}}

\hlstd{mean_loss_ave} \hlkwb{<-} \hlkwa{function}\hlstd{(}\hlkwc{x}\hlstd{,} \hlkwc{theta} \hlstd{=} \hlnum{1}\hlstd{) \{}
    \hlkwd{mean}\hlstd{(}\hlkwd{mean_loss}\hlstd{(x, theta))}
\hlstd{\}}

\hlstd{compare_eq_1} \hlkwb{<-} \hlkwa{function}\hlstd{(}\hlkwc{n_sample}\hlstd{,} \hlkwc{theta} \hlstd{=} \hlnum{1}\hlstd{) \{}
    \hlstd{mu_x} \hlkwb{<-} \hlkwd{ifelse}\hlstd{(theta} \hlopt{==} \hlnum{1}\hlstd{,} \hlopt{-}\hlkwd{qnorm}\hlstd{(}\hlnum{0.06}\hlstd{),} \hlopt{-}\hlkwd{qnorm}\hlstd{(}\hlnum{0.98}\hlstd{))}
    \hlstd{x_star} \hlkwb{<-} \hlkwd{rnorm}\hlstd{(n_sample, mu_x,} \hlnum{1}\hlstd{)}
    \hlstd{lhs} \hlkwb{<-} \hlkwd{r_cond_ave}\hlstd{(x_star, theta)}
    \hlstd{rhs} \hlkwb{<-} \hlkwd{mean_loss_ave}\hlstd{(x_star, theta)}
    \hlkwd{return}\hlstd{(}\hlkwd{c}\hlstd{(lhs, rhs))}
\hlstd{\}}

\hlstd{get_table} \hlkwb{<-} \hlkwa{function}\hlstd{(}\hlkwc{n_sample}\hlstd{) \{}
    \hlstd{theta_1} \hlkwb{<-} \hlkwd{compare_eq_1}\hlstd{(n_sample,} \hlnum{1}\hlstd{)}
    \hlstd{theta_2} \hlkwb{<-} \hlkwd{compare_eq_1}\hlstd{(n_sample,} \hlnum{2}\hlstd{)}
    \hlstd{df} \hlkwb{<-} \hlkwd{as.data.frame}\hlstd{(}\hlkwd{rbind}\hlstd{(theta_1, theta_2))}
    \hlkwd{colnames}\hlstd{(df)} \hlkwb{<-} \hlkwd{c}\hlstd{(}\hlstr{"$E\textbackslash{}\textbackslash{}\{l_i(s(X_i))| \textbackslash{}\textbackslash{}theta(i)\textbackslash{}\textbackslash{}\}$"}\hlstd{,} \hlstr{"$E\textbackslash{}\textbackslash{}\{l(\textbackslash{}\textbackslash{}theta(i), s(X_i))|\textbackslash{}\textbackslash{}theta(i)\textbackslash{}\textbackslash{}\}$"}\hlstd{)}
    \hlkwd{rownames}\hlstd{(df)} \hlkwb{<-} \hlkwd{c}\hlstd{(}\hlstr{"$\textbackslash{}\textbackslash{}theta(i)=1$"}\hlstd{,} \hlstr{"$\textbackslash{}\textbackslash{}theta(i)=2$"}\hlstd{)}
    \hlkwd{return}\hlstd{(df)}
\hlstd{\}}

\hlstd{results} \hlkwb{<-} \hlkwd{get_table}\hlstd{(}\hlnum{1000}\hlstd{)}
\end{alltt}
\end{kframe}
\end{knitrout}


The simulation results are summarized in the following table.

% latex table generated in R 3.0.2 by xtable 1.7-1 package
% Thu Feb  6 14:59:18 2014
\begin{table}[ht]
\centering
\begin{tabular}{rrr}
  \hline
 & $E\{l_i(s(X_i))| \theta(i)\}$ & $E\{l(\theta(i), s(X_i))|\theta(i)\}$ \\ 
  \hline
$\theta(i)=1$ & 2.23 & 2.16 \\ 
  $\theta(i)=2$ & 0.07 & 0.02 \\ 
   \hline
\end{tabular}
\end{table}


\end{proof}

\subsection*{Problem 3}
Assume that the random variable $X$ has finite $E|X|$ and is continuous (has a density). Show that $E|X - a|$ is minimun at $a=$ median$(X)$.
\begin{proof}
Let $f(x)$ be the density function of $X$. Calculating,
\begin{align*}
\frac{d}{da} E|X-a| &= \frac{d}{da} \int_{-\infty}^{\infty} |x - a| f(x) dx \\
&= \int_{-\infty}^{\infty} \frac{d}{da}  |x - a| f(x) dx 
\qquad \text{by \href{http://en.wikipedia.org/wiki/Leibniz_integral_rule}{Liebniz's Rule}}\\
&= \int_{-\infty}^{a} f(x) dx + \int_{a}^{\infty} - f(x) dx \\
&= \int_{-\infty}^{a} f(x) dx - \int_{a}^{\infty} f(x) dx \\
&= F(a) - (1 - F(a)) \\
&= 2 F(a) - 1
\end{align*}
Setting this derivative to $0$, we have equality for the value of $a$ for which $F(a) = 0.5$, or when $a =$ median$(X)$. The second derivative with respect to $a$ is $\frac{d}{da}(2F(a) - 1) = 2f(a) >0$. Thus the $a$ that we found is a minimum for the original function.
\end{proof}

\subsection*{Problem 4}
We want to estimate the true value of the scalar $\theta$, and we have a loss function $l(\theta, a) = |\theta - a|$. Based on previous similar studies, we believe that, a priori, $P(\theta) = N (\mu_0, \tau_0^2)$, where $\mu_0$ and $\tau_0$ are known values. To help us estimate $\theta$, we design a study that gives us data $X$ where 
$P(X | \theta) = N(\theta, \sigma_0^2)$, and  where $\sigma_0^2$ is assumed known.
\subsubsection*{Part 1}
Find the posterior distribution $P(\theta | X)$.
\begin{proof}
As a lemma, let the likelihood $\pr(X \sbs \theta) = N_k(\theta, \Sigma)$ be multivariate normal with prior $\pr(\theta) = N_k(\mu, V)$ multivariate normal. The posterior $\pr(\theta \sbs X)$ is proportional to the product of the likelihood and the prior. Hence 
\begin{align}
\pr(\theta \sbs X) &\propto
\exp\left[-\frac{1}{2}(x-\theta)'\Sigma^{-1}(x-\theta) - 
\frac{1}{2}(\theta - \mu)'V^{-1}(\theta-\mu)\right]
\\
&\propto
\exp\left[
-\frac{1}{2}(\theta'\Sigma^{-1}\theta + \theta'V^{-1}\theta -2\theta'\Sigma^{-1}x -2\theta'V^{-1}\mu)
\right]
\\
&=
\exp\left[
-\frac{1}{2}
(\theta'(\Sigma^{-1} + V^{-1}) \theta
-2\theta'(\Sigma^{-1}x +V^{-1}\mu) )
\right]
\end{align}
Therefore, the posterior is proportional to an unnormalized multivariate normal distribution with 
\begin{equation}
N_k((\Sigma^{-1} + V^{-1})^{-1}(\Sigma^{-1}x +V^{-1}\mu), (\Sigma^{-1} + V^{-1})^{-1})
\end{equation}
For this problem, there are multiple observations $X$ and a common mean $\theta$. Notice
\begin{align}
\pr(x_1, x_2, \dots, x_n \sbs \theta) &\propto_\theta
\exp\left[
-\frac{1}{2 \sigma_0^2}
\sion(x_i - \theta)^2
\right]
\\
&\propto_\theta
\exp\left[
-\frac{1}{2 \sigma_0^2}
\sion(x_i^2 - 2x_i\theta + \theta^2)
\right]
\\
&\propto_\theta
\exp\left[
-\frac{n}{2 \sigma_0^2}
\sion(2\bar{x_n}\theta + \theta^2)
\right]
\\
&\propto_\theta
\exp\left[
-\frac{n}{2 \sigma_0^2}
\sion(\bar{x_n} - \theta)^2
\right]
\end{align}
which is recognized as the normal distribution $\pr(\bar{x_n}|\theta)=N(\theta, \frac{\sigma_0^2}{n})$. Calculating more,
\begin{align}
\pr(\theta \sbs x_1, x_2, \dots, x_n) &\propto_\theta 
\pr(x_1, x_2, \dots, x_n \sbs \theta) \pr(\theta)
\\
&\propto_\theta
\pr(\bar{x_n}|\theta) \pr(\theta)
\\
&\propto_\theta
\pr(\theta \sbs \bar{x_n})
\end{align}
which according to the lemma, follows the
\begin{equation}
N
\left(
\left(
\frac{n}{\sigma_0^2} + \frac{1}{\tau_0^2}
\right)^{-1}
\left(
\frac{n}{\sigma_0^2}\bar{x_n} + \frac{1}{\tau_0^2}\mu_0
\right), 
\left(
\frac{n}{\sigma_0^2} + \frac{1}{\tau_0^2}
\right)^{-1}
\right)
=
N
\left(
\frac{\tau_0^2}{\frac{\sigma_0^2}{n}+ \tau_0^2} \bar{x_n} +
\frac{\sigma_0^2}{\sigma_0^2+ n\tau_0^2} \mu_0,
\left(
\frac{n}{\sigma_0^2} + \frac{1}{\tau_0^2}
\right)
^{-1}
\right)
\end{equation}
Source: \href{http://www.cs.berkeley.edu/~jordan/courses/260-spring10/lectures/lecture5.pdf}{Berkeley course notes}
\end{proof}
\subsubsection*{Part 2}
Using Exercise 3, find the Bayes estimator for this problem, i.e. the estimate $s(X)$ that minimizes $E \{E (l ( \theta, s(X)) \sbs \theta  ) \}$, where the outer expectation is with respect to the prior distribution for $\theta$.
\begin{proof}
According to the class notes, slide notes (Chapter 1, pg. 20),
\begin{equation}
E_\theta [E_{X\sbs\theta} \{l ( \theta, s(X)) \sbs \theta  \} ]=
E_{X} [E_{\theta\sbs X} \{l ( \theta, s(X)) \sbs X  \} ]
\end{equation}
The $s(X)$ is chosen that minimizes the posterior loss
\begin{equation}
E_{\theta\sbs X} \{l ( \theta, s(x)) \sbs X =x \}
=
\int_\theta 
l(\theta, s(x)) \pr(\theta \sbs X=x)d\theta
\end{equation}
To do that, first the ``no-data problem'' is solved. Assume a working prior $\pi(\theta)$. The $s^*$ that minimizes 
\begin{equation}
\int_\theta 
l(\theta, s^*) \pi(\theta)d\theta = E_\pi [|\theta - s^*|]
\end{equation}
is $s^*_{(\pi)} = \text{median}(\theta)$ by problem 3. Therefore, by slide notes (Chapter 1, pg. 21), the Bayes estimator is
\[
s^*(X) = s^*_{\pr(\theta \sbs X)} = \text{median}(\theta) = \frac{\tau_0^2}{\frac{\sigma_0^2}{n}+ \tau_0^2} \bar{x_n} +
\frac{\sigma_0^2}{\sigma_0^2+ n\tau_0^2} \mu_0
\]
\end{proof}

\subsection*{Problem 5}
Refer to Problem 4, and suppose we have iid observations from the likelihood. By considering a sequence of priors, each as in problem 4, but with mean $0$ and $\tau_0$ increasing with the sequence, show that the sample average is a minimax estimator. (Hint: note that the sample average is equalizer).

\begin{proof}
Let $\{\tau_j\} \uparrow \infty$ be a sequence increasing to infinity. Let $\theta_j$ be distributed as $N(0, \tau_j^2)$. Then the posterior distribution $\pr(\theta_j | x_1, x_2, \dots, x_n)$ is 
\[
N\left(
\frac{\tau_j^2}{\frac{\sigma_0^2}{n} + \tau_j^2} \bar{x_n}, 
\left(
\frac{n}{\sigma_0^2} + \frac{1}{\tau_j^2}
\right)^{-1}
\right)
\]
It is easy to see the limit of the posterior is 
\[
\lim_{j\to \infty} \pr(\theta_j\sbs x_1, x_2, \dots, x_n) 
= 
N
\left(
\bar{x_n}, \frac{\sigma_0^2}{n}
\right)
\]

%%%%%%%%%%%%%%%%

We will use the following result to prove the sample average is minimax.

\textbf{Result 1:} If $s_j$ is a Bayes rule with respect to $\pi_j$, with $L(\theta, s_j)\to c\in \R$ and if there exists a strategy $s_0$ such that $L(\theta, s_0) \leq c$ for every $\theta \in \Theta$, then $s_0$ is minimax with ``value''
\[
\inf_s \sup_\theta L(\theta, s) = c = \sup_\theta L(\theta, s_0)
\] 

Suppose $s|\theta \sim N(\mu, \sigma^2)$. Calculating,
\begin{align}
E_{s|\theta}[l(\theta, s)] 
&=
\int_{-\infty}^\infty |\theta - s| f_{s|\theta}(s) ds
\\
&= 
\int_{-\infty}^\theta (\theta - s) f_{s|\theta}(s) ds +
\int_\theta^\infty (s - \theta) f_{s|\theta}(s) ds
\\
&=
\int_{-\infty}^\theta (\theta -\mu + \mu - s) f_{s|\theta}(s) ds +
\int_\theta^\infty (s - \mu +\mu- \theta) f_{s|\theta}(s) ds
\\
&=
(\theta - \mu)F_{s|\theta}(\theta) + (\mu - \theta)(1-F_{s|\theta}(\theta) ) + 
\int_{-\infty}^\theta (\mu - s) f_{s|\theta}(s) ds +
\int_\theta^\infty (s - \mu) f_{s|\theta}(s) ds
\\
&=
(\theta - \mu)(2 F_{s|\theta}(\theta) - 1)
+
\int^{\infty}_{-\theta} (s - \mu) f_{s|\theta}(s) ds +
\int_\theta^\infty (s - \mu) f_{s|\theta}(s) ds
\end{align}
And doing a $u$ substitution with $u = \frac{s - \mu}{\sigma}$
\begin{align}
\int^{\infty}_{-\theta} (s - \mu) f_{s|\theta}(s) ds +
\int_\theta^\infty (s - \mu) f_{s|\theta}(s) ds
&=
\frac{\sigma}{\sqrt{2\pi}}
\left(
\int_{-\frac{\theta - \mu}{\sigma}}^\infty
u \exp\{-0.5u^2\}
+
\int_{\frac{\theta - \mu}{\sigma}}^\infty
u \exp\{-0.5u^2\}
\right)
\\
&=
\frac{\sigma}{\sqrt{2\pi}}
\left(
2 \exp\left\{
-\frac{1}{2\sigma^2} (\theta-\mu)^2
\right\}
\right)
\end{align}
Hence
\begin{equation}
E_{s|\theta}[l(\theta, s)] = (\theta - \mu)(2 F_{s|\theta}(\theta) - 1) + \frac{2\sigma}{\sqrt{2\pi}}
 \exp\left\{
-\frac{1}{2\sigma^2} (\theta-\mu)^2
\right\}
\end{equation}
Note that $E_{s|\theta}[l(\theta, s)] >0 $ when $|\theta - \mu| > 0$. 

Since
\[
s_{\pi_j}(X) = \frac{\tau_j^2}{\frac{\sigma_0^2}{n} + \tau_j^2} \bar{x_n}
\]
we have that 
\[
s_{\pi_j}(X) | \theta 
\sim 
N
\left(
\frac{\tau_j^2}{\frac{\sigma_0^2}{n} + \tau_j^2} \theta
,
\frac{\sigma_0^2}{n} \left(\frac{\tau_j^2}{\frac{\sigma_0^2}{n} + \tau_j^2}\right)^2
\right)
\]
Using the result above,
\begin{equation}
L(\theta, s_{\pi_j}) = 
(
\theta - \frac{\tau_j^2}{\frac{\sigma_0^2}{n} + \tau_j^2} \theta
)
(2 F_{s_{\pi_j}|\theta}(\theta) - 1) 
+
\frac{2\sigma_0}{\sqrt{2\pi n}}
\left(\frac{\tau_j^2}{\frac{\sigma_0^2}{n} + \tau_j^2}\right)
\exp
\left\{
-\frac{n}{2\sigma_0^2}
\left(\frac{\frac{\sigma_0^2}{n} + \tau_j^2}{\tau_j^2}\right)
\left(\theta
-
\frac{\tau_j^2}{\frac{\sigma_0^2}{n} + \tau_j^2} \theta
\right)^2
\right\}
\end{equation}
Taking the limit,
\[
\lim_{j \to \infty} L(\theta, s_{\pi_j}) = \frac{2\sigma_0}{\sqrt{2\pi n}}
\]
Now $L(\theta, \bar{x_n}) = E_{\bar{x_n}|\theta}[l(\theta, \bar{x_n})]$ and since 
\[
\bar{x_n} | \theta \sim 
N
\left(
\theta, \frac{\sigma_0^2}{n}
\right)
\]
it follows
\[
L(\theta, \bar{x_n}) = 
\frac{2\sigma_0}{\sqrt{2\pi n}}
\]
That is $\bar{x_n}$ is equalizer.
Therefore, we have the conclusion of the \textbf{Result 1} above, i.e. that the sample average is equalizer.
\end{proof}



\end{document}
