\documentclass[letterpaper,12pt]{article}

\usepackage{amsmath,amsfonts,amsthm,amssymb}
\usepackage[margin=1in]{geometry}

\usepackage{enumerate}
\usepackage{hyperref}

\renewcommand{\P}{\mathcal{P}}

\begin{document}

\textsf{
\begin{flushleft}
\sc James K. Pringle \\
\normalfont 140.673 \\
Dr. Constantine Frangakis \\
Assignment 1 \\
7 February 2013, Thursday
\end{flushleft}
} \bigskip

\begin{center}
\bf Problems 1, 2, 3, 4, 5
\end{center}

\begin{enumerate}[(1)]
\item
Among the population, $P$, of women who visit phyisicians for screening for a disease, assume that the screening test has specificity and sensitivity as the one discussed in lecture one, where, here, probability statements mean fractions of women in the population $P$ (for example, specificity of $98\%$ means that, of all true negative women in $P$, $98\%$ would test negative.).
For a woman $i$, denote $\theta(i) = 1$ if the woman is truly positive, and $2$ if truly negative; and denote $(l_i(a_1), l_i(a_2))$ to be the loss to that woman if treated, or if not treated, respectively (in the last expressions, her true status is captured already in the notation ``i'').
Suppose that the averages of the losses in the diseased and non-diseased women, if treated and if not treated, are:
\begin{align*}
l(1,a_1) &:= E\{l_i(a_1)| \theta(i) = 1\} = 2 \\
l(1, a_2) &:= E\{l_i(a_2)|\theta(i)=1\} = 5 \\
l(2, a_1) &:= E\{l_i(a_1)|\theta(i)=2\} = 1 \\
l(2, a_2) &:= E\{l_i(a_2)|\theta(i)=2\} = 0
\end{align*}
but that $l_i$ may generally vary from woman to woman, and that among women of a particular status (diseased, or not diseased), the losses $l_i$ may be correlated with the value $X_i$ that the diagnostic test would show for woman $i$.
For the strategy $s$ defined as $s(X_i) = a_1$ if $X_i$ is positive, and $s(X_i)=a_2$ if $X_i$ is negative, which of the following conditions 1.-3. would make the losses
\begin{equation} 
\label{eq1}
E\{l_i(s(X_i)) | \theta(i)\} \text{ and } E\{l(\theta(i),s(X_i))| \theta(i)\}
\end{equation}
equal and why?
\begin{enumerate}[1.]
\item
For a fixed action $a$, $l_i(a)$ is constant within women of common disease status $\theta(i)$.
\item
For a fixed action $a$, $X_i$ is independent 	of $l_i(a)$.
\item
For a fixed action $a$, $X_i$ is independent of $l_i(a)$ given $\theta(i)$.
\end{enumerate}

\begin{proof}
Let's examine 1. For an action $a$, the loss $l_i(a)$ is constant within women of common disease status $\theta(i)$. Since, for example,  $E\{l_i(a_1)| \theta(i) = 1\} = 2$, it must be that $l_i(a_1)$ given $\theta(i) =1$ is $2$ for all $i$. Otherwise, if $l_i(a_1)$ were some other value, $c \neq 2$, then the expected value would be $c$. In other words, we have $l_i (s(X_i)) = l(\theta(i), s(X_i))$, since $s(X_i)$ is the same on both sides of the equal sign and $\theta(i)$ is captured in $l_i$. Since we have that equality, taking expectations over subsets of people with true status $\theta(i)$ will yield equal values.

Now 2. And 3.
\end{proof}

\item
Now assume that the way the test $X_i$ is determined is by measuring a continuous variable $X_i^*$, and calling $X_i$ positive if $X^* > 0$, otherwise calling $X_i$ negative. Assuming that pr$(X_i^* | \theta(i))$ is normal with variance $1$, find $E(X_i^*|\theta(i))$ for the two disease conditions. 
Also, assume that, for a fixed action $a$, pr$(l_i(a) | \theta(i))$ is normal with the means given above, variance $10$, and that $cor(l_i(a),X_i^*|\theta(i))=0.7$.
 Using simulation of $1000$ 	diseased and $1000$ non-diseased women, or otherwise, estimate the two average losses in \eqref{eq1}.

\begin{proof}
From the lecture notes, the specificiy of the test is $0.98$ and the sensitivity is $0.94$. Thus $P( X_i^* > 0 | \theta(i) = 1) = 0.94$. Since we assume a normal distribution, with variance $\sigma^2 = 1$, we have
\begin{align*}
0.94 &= P( X^* > 0 | \theta(i) = 1) \\
0.94 &= P( (X^* - \mu_1) / \sigma > -\mu_1/ \sigma | \theta(i) = 1) \\
0.94 &= P( Z > -\mu_1 | \theta(i) = 1)
\end{align*}
Thus we solve for $\mu_1 = E(X_i^* | \theta(i)=1)$ in $1 - F_Z(-\mu_1) = 0.94$ where $F_Z$ is the distribution function for the standard normal distribution. It follows that $\mu_1 = -F_Z^{-1} (0.06) = - \texttt{qnorm}(0.06) = 1.554774$. 

Similarly solving for $\mu_2 = E(X_i^* | \theta(i) = 2)$, one finds that
\begin{align*}
0.98 &= P(X^* < 0 | \theta(i) = 2) \\
0.98 &= P((X^* - \mu_2) \ \sigma < -\mu_2 / \sigma | \theta(i) = 2) \\
0.98 &= P( Z < -\mu_2 | \theta(i) = 2)
\end{align*}
So $\mu = -F_Z^{-1}(0.98) = -\texttt{qnorm}(0.98) = -2.053749$.

The simulations yielded $2.184648$ for the first expectation and $2.156$ for the second expectation.
\end{proof}

\item
Assume that the random variable $X$ has finite $E|X|$ and is continuous (has a density). Show that $E|X - a|$ is minimun at $a=$ median$(X)$.
\begin{proof}
Let $f(x)$ be the density function of $X$. Calculating,
\begin{align*}
\frac{d}{da} E|X| &= \frac{d}{da} \int_{-\infty}^{\infty} |x - a| f(x) dx \\
&= \int_{-\infty}^{\infty} \frac{d}{da}  |x - a| f(x) dx \\
&= \int_{-\infty}^{a} f(x) dx + \int_{a}^{\infty} - f(x) dx \\
&= \int_{-\infty}^{a} f(x) dx - \int_{a}^{\infty} f(x) dx
\end{align*}
Setting this derivative to $0$, we have equality for the value of $a$ for which
\[
F(a) = \int_{-\infty}^{a} f(x) dx = \int_{a}^{\infty} f(x) dx = 1 - F(a) \text{.}
\]
That is where $1 - F(a) = F(a) = 0.5$, or when $a =$ median$(X)$. The second derivate with respect to $a$ is $d/da(F(a) - (1 - F(a))) = 2f(a) >0$. Thus the $a$ that we found is a minimum for the original function.
\end{proof}

\item
We want to estimate the true value of the scalar $\theta$, and we have a loss function $l(\theta, a) = |\theta - a|$. Based on previous similar studies, we believe that, a priori, $\P(\theta) = N (\mu_0, \tau_0^2)$, where $\mu_0$ and $\tau_0$ are known values. To help us estimate $\theta$, we design a study that gives us data $X$ where 
 $\P(X | \theta) = N(\theta, \sigma_0^2)$, and  where $\sigma_0^2$ is assumed known.
\begin{enumerate}[1.]
\item
Find the posterior distribution $\P(\theta | X)$.
\begin{proof}
According to \url{http://zoe.bme.gatech.edu/~bv20/bmestat/Bank/bayes.pdf}, the posterior distribution is 
\[
\theta | X \sim N \left( \frac{\tau_0^2}{\sigma_0^2 + \tau_0^2} X + \frac{\sigma_0^2}{\sigma_0^2 + \tau_0^2} \mu_0, \frac{\sigma_0^2\tau_0^2}{\sigma_0^2 + \tau_0^2} \right)
\]
\end{proof}
\item
Using Exercise 3, find the Bayes estimator for this problem, i.e. the estimate $s(X)$ that minimizes $E \{E (l ( \theta, s(X)) | \theta  ) \}$, where the outer expectation is with respect to the prior distribution for $\theta$.
\end{enumerate}

\item
Refer to Problem 4, and suppose we have iid observations from the likelihood. By considering a sequence of priors, each as in problem 4, but with mean $0$ and $\tau_0$ increasing with the sequence, show that the sample average is a minimax estimator. (Hint: note that the sample average is equalizer).

\end{enumerate}

\end{document}